\documentclass[A4]{scrreprt}
\usepackage[paper=a4paper,left=10mm,right=10mm,top=25mm,bottom=25mm]{geometry}
\usepackage[utf8]{inputenc}
\usepackage{graphicx}

\begin{document}
  \addchap{Wärmelehre}
  \addsec{Grundlagen}
    Mol: $1 mol = N_A Atome = 6.02\cdot 10^{23} Teilchen$\\
    Atomar mass unit: $1u = 1.66053 \cdot 10^{-27} kg$\\
    Größe Atom: $\approx 1\cdot 10^{-10} m$ \\
  \addsec{Thermodynamik Idealer Gase}
    Ideale Gase:\\
    Boltzmann Konstante: $k_B = 1,38 \cdot 10^{-23} \frac{kg \cdot m^2}{s^2 \cdot K}$\\
    mittlere quadratische Geschindigkeit: $\frac{1}{2} k_B \cdot T$
    Abstand von Atomen unter Normaldruck: $5\cdot 10^{-9} m$\\
    Mittlere Geschwindigkeit: $v = 0 \frac{m}{s}$\\
    Mittlere Kinetische Energie: $E = \frac{3}{2}m_{teilchen}v^2$\\
    Allg. Gasgleichung: $p\cdot V = N \cdot K_B \cdot T$\\
    Teilchendichte: $ n = \frac{N}{V}$\\
    Druck: $p = \frac{F}{A}$ \\
    Druck in einer Flüssigkeit: $p(h) = \rho \cdot h \cdot g$\\
    Gleichverteilungssatz: $E_{kin} = E_{tra} + E{rot} = \frac{f}{2} \cdot K_B \cdot t$\\
    Barom. Höhenformel: $p(h) = p_0 \cdot e^{-\frac{h}{h_0}}$\\
    Energiesatz: $\Delta U = \Delta Q + \Delta W$\\
    Innere Energie: $U = \frac{f}{2} K_B \cdot T \cdot N$\\
    Isentropenkoeffizient: $\kappa = 1+ \frac{2}{f}$
    \\ 
    \begin{tabular}{l|l}
    Isotherme Zustandsänderung:&Isochore Zustandsänderung:\\\\
    T = const&V = const;\\
    $\Delta W = -NK_bT ln(\frac{p1}{p2})$&$\Delta W = 0$\\
    $\Delta Q = -\Delta W = NK_bT ln(\frac{p1}{p2})$&$\Delta Q = \frac{f}{2}NK_b\Delta T = \frac{f}{2}\cdot V \Delta p$\\
    $\Delta U = 0$&$\Delta U = \frac{f}{2}NK_b\Delta T = \frac{f}{2}\cdot V \Delta p$\\
    $\Delta S = NK_bln(\frac{p-1}{p_2})$&$\Delta S = \frac{f}{2} NK_b \rightarrow const.$\\\\

    Isobare Zustandsänderung:&Adiabate Zustandsänderung:\\
    p = const;&$p\cdot V^{\kappa} = const$\\
    $\Delta W = -p dV = NK_b \Delta T$&$\Delta W = p_1V_1\frac{f}{2}((\frac{V_2}{V_1})^{\frac{f}{2}}-1)$\\
    $\Delta Q = (\frac{f}{2}-1)p \Delta V$&$\Delta Q = 0$\\
    $\Delta U = \frac{f}{2}NK_b \Delta T = \frac{f}{2}p \Delta V$&$\Delta U = 0$\\
    $\Delta S = (\frac{f}{2}-1)p \frac{\Delta V}{T}$&$\Delta S = 0$\\
    \end{tabular}\\
    \\\\
    Wirkungsgrad: $\eta = \frac{P_{ab}}{P_{zu}} = \frac{W_{ab}}{W_{zu}} = \frac{|\Delta W|}{|\Delta Q_h|}$\\
    Carnot Wirkungsgrad: $\frac{T_h - T_t}{T_h}$\\
    Entropie: $S = K_bln(W)$,$\Delta S = K_bln(\frac{W_2}{W_1}) = \frac{\Delta Q}{T} $\\
    Satz v. Stirling: $ln(N!) = N ln(N)$ für $N>>0$
  \addsec{Thermodynamik Realer Gase}
    Van der Waals Gleichung: $NK_bT=(P+aN^2\frac{1}{V^2})\cdot(V-Nb)\rightarrow p = \frac{NK_bT}{V-Nb}-\frac{N^2 a}{V^2}$
    
  \addsec{Wärmeleitung und Wärmeausdehnung}
  Wärmekapazität v = const: $c_K=\frac{f}{2}\cdot N_A K_B$\\
  Wärmekapazität p = const: $c_K=(\frac{f}{2}+1)N_A K_B$\\
  Wärmestrom: $Q'_{(t)}=q'A = -\lambda\cdot\frac{dT}{dx}$\\
  Dulong-Petit'sche Regel: $c_K = 3 N_A K_B$\\
  Wärmeausdehnung linear: $\Delta L = \alpha L \Delta T$\\
  Wärmeausdehnung Kubisch: $\Delta V = \beta V \Delta T , \beta = 3\alpha$\\
  Richmannsche Mischungsregel: TODO\\
  Richmannsche Mischungsregel mit Wärmemengen: TODO\\

  
  \addchap{Wellen und Optik}
  \addsec{Wellengleichung}
  Allg. Wellengleichung:\\
  Lsg für ebene Wellen im $R^1$: $A_{(x,t)} = A_0\cdot cos(\omega t - k*x_0)$, $k = \frac{2\pi}{\lambda}$, $\omega = \frac{2\pi}{T}$\\
  Lsg für ebene Wellen im $R^3$: \\
  Lsg für Kreiswellen: $A(x,y,t) = \frac{A_0}{\sqrt{r}}\cdot cos(\omega t - |k|\cdot\sqrt{x^2+y^2})$\\
  Lsg für Zylinderwellen: \\
  Lsg für Kugelwellen: $A(x,y,t) = \frac{A_0}{r}\cdot cos(\omega t - |k|\cdot \sqrt{x^2+y^2+z^2})$\\
  \\
  Wellenlänge: $\lambda = c\cdot T$\\  

  \addsec{Interferenz}
  Interferenz:\\

  \addsec{Doppler Effekt}
  Quelle ruht, Empfänger bewegt: $f_E = f_Q (1-\frac{v_E}{c})$\\
  Empfänger ruht, Quelle bewegt: $f_E = f_Q\cdot(\frac{1}{1-\frac{V_Q}{c}})$\\
  Beide bewegt: $f_e=f_s\frac{c-{V_s}}{c-{V_e}}$

  \addsec{}
\end{document}
% 1 Oktave = Verdopplung der Frequenz = 12 Halbtöne
% Frequenzverhältnis der Halbtöne ist konstant
% f' -> f": f''/f' = f'''/f'' = x
% x^12 = 2 -> x = 2^{1/12}
%Brechungsindex: duch Transportschicht verursachter Gangunterschied
% 1: keine Brechung
% : 
%Transmission
