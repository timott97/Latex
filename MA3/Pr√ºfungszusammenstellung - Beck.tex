\documentclass[A4]{scrartcl}

\begin{document}
\addsec {Differentialgleichungen}
$y'+2x\cdot y^2 = 0 $\\
Berechnen sie die allgemeine Lösung der obigen DGL.\\
Geben sie die spezielle Lösung für das AWP $y(0) = -1 $ an und skizzieren Sie die Lösung für $-3 \leq x \leq 3$. \\\\
$y' + \frac{x}{y} = 0$
Berechnen sie die allgemeine Lösung der DGL
Skizzieren Sie die Schar der lösungskurven. Welche geometrische Form beschreibt jede der Lösungskurven?\\\\
$x \cdot y' + y = x \cdot cos(x)$\\
Berechnen sie die allgemeine Lösung der DGL durch Variation der Konstanten!\\\\
Geben sie die spezielle Lösung der obigen DGL für folgende Randbedingungen an: $y(\pi ) = 0$\\
Führen sie eine Probe durch und zeigen Sie, dass die von Ihnen gefundene Lösung tatsächlich eine Lösung der DGL ist.\\\\
$y''+3y'-4y = sin (x)$\\
Berechnen sie die allgemeine Lösung der DGL mit Bethoden, die in der Vorlesung MA3 verwendet wurden.\\\\
Berechnen sie die Spezielle Lösung der obigen DGL für folgende Anfangsbedingungen:
$y(0)=-\frac{3}{34}$,$y'(0) = -\frac{5}{34}$
wie groß ist die Phasenverschiebung zwischen der speziellen Lösung und der Störfunktion $sin(x)$?\\\\
$y'x-2y=x^3 \cdot cos(4x)$\\
Berechnen sie die allgemeine Lösung der DGL durch Variation der Konstanten. \\\\
$y' = 2 \cdot \frac{y^2}{x^2}+\frac{y}{x}$\\
Berechnen sie die allgemeine Lösung der DGL mit Hilfe einer geeigneten Substitution.\\\\
$y''+2y'-3y-x=0$\\
Berechnen sie die allgemeine Lösung der DGL.\\\\
$xy''+y = x^3 +6x^2 +2x$\\
Lösen sie die DGL mit den Anfangsbedingungen $y(0) = 0 $ und $y'(0) = 2$ mit Hilfe des Potenzreihenansatzes.\\
Berechnen sie die Potenzreihe bis zur Ordnung $x^3$.\\
Überprüfen sie, ob das berechnete Polynom die DGL löst.\\\\
$y'' + 2y' - 3y = 2cos(t)-4sin(t)$\\
Berechnen sie die allgemeine Lösung der DGL\\\\
$x^3 + (y+1)^2 y' =0$ mit $y(0) =0$\\
Berechnen sie die Lösung der DGL.\\\\
$y'=(x+y)^2$\\
Berechnen sie die allgemeine Lösung der DGL 
(Hinweis: Die Stammfunktion der Funktion $f(x) = \frac{1}{x^2 +1}$ ist $F(x) = arctan(x)+C$\\\\
$y'- \frac{y}{x} = x$\\
Berechnen sie die allgemeine Lösung der DGL.

TODO: Numerische LSG, Euler Verfahren
\newpage

\addsec {Matrizen}

\addsec {Scilab}
Gegeben sei folgender Scilab-Code:\\
function dydx = f(x,y);\\
dydx=x;\\
endfunction;\\
y0=2;x0=0;\\
y=ode(y0,x0,x,f);\\
plot2d(x,y)\\
c=y(5)\\\\
a) Skizzieren Sie die Kurve die Scilab erstellt, in ein Koordinatensystem.\\
b) Welchen Wert gibt Scilab für die Variable c aus?\\
\end{document}